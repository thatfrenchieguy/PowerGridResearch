\chapter{Conclusions}
\index{Conclusions@\emph{Conclusions}}%

	\section{Conclusions}

As discussed in the literature review, interaction between actors in a repair context has not been thoroughly explored. We present a pair of models and analyze several perturbations of standard IEEE test grids to demonstrate the effectiveness of interacting the models in several different frameworks. This yields a series of results that are closer to a theoretical lower bound as compared to treating repairs as a pure scheduling problem on the power grid and applying routing as an after-the-fact post processing step as is done in previous modeling efforts that generate only a schedule and leave routing to the agency conducting repairs.

These repair models are then extended into resilience models to show that interdiction based modeling performs better than a heuristic method at minimizing the unsatisfied power demand over a basket of random damage instances. This suggests that further consideration of multiple network layers can lead to better insights when considering resilience planning of multiple network layers.


\section{Future Research Directions}
The natural extensions for future research are to take the models outlined and fit them to the topology of a real place that is struck by hurricanes such as Houston and then simulate a hurricane strike to generate the damage scenario. This would require an involved effort to correctly model both flooding/storm surge as well as wind damage, but is possible. Additionally along the same line of research, treating the repair problem outlined above as a recourse step in a two stage stochastic program based on a suite of hurricane scenarios would be an interesting direction of study. The first step could take the form of inventory location and quantity or problem about network hardening.

Another research direction that could be undertaken is to account for imperfect information about the state of the power grid or imperfect information about the status of the road network. Sharing of resources (e.g space on a truck moving supplies in) as a form of optimization under uncertainty which would also have implications for interesting interactions between decision makers in the repair effort. Along those lines, optimization of roads has implications for other types of network infrastructure such as water supplies and rail/mass transit networks.