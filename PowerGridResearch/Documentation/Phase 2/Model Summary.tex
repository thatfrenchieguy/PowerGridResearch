\documentclass{article}
\usepackage{amsmath}
\usepackage[fleqn]{mathtools}
\usepackage{amssymb}
\usepackage{amsthm}
\usepackage{enumitem}

\begin{document}
	\title{Model Description for the Phase 2 Hurricane Recovery Problem}
	\author{Brian French}
	\maketitle
	
	\section{Variable Glossary}
	\subsection{Constants}
	\begin{itemize}
		\item $C_{FlowI}$ is steady state flow on the grid before the hurricane
		\item $C_{lineIJ}$ is the capacity limit for the power line going from IJ
		\item $C_{RepairTimeI}$ is the time to repair node I
		\item $C_{TravelIJ}$ is the travel time between nodes I and J
		\item $C_{broken}$ is a coefficient of "broken-ness" representing the average slowdown from debris on the road and minor flooding
	\end{itemize}
	\subsection{Variables}
	\begin{itemize}
		\item $Z_{i}^{t}$ is the total power flow at node i at time t
		\item $X_{ij}^{t}$ is the flow from i to j at time t
		\item $Y_i^t$ is 1 if node i is functioning at time t
		\item $W_{ij}^t$ is 1 if line ij is functioning at time t
		\item $S_{i}^t$ is 1 if node i is serviced at time t
		\item $K_{ij}^t$ is 1 if node j follows node i in the tour at time t 
		\item $F_i^t$ is 1 is node i is serviced at time t t
	\end{itemize}
\subsection{Sets}
\begin{itemize}
	\item L is the set of nodes
	\item P is the set of power lines
	\item R is the set of roads
	\item T is the planning horizon
\end{itemize}
	\section{Model}
	$$	Minimize \sum_{i \in L} \sum_{t \in T} C_{FlowI}-Z_{i}^t $$
	
	Subject to:
	\begin{itemize}
		\item $Z_i^t = (\sum_{j \in L} X_{ji}^t) Y_i^t \hspace{4pt} \forall t \in T \hspace{4pt} \forall i \in L$
		\item $\sum_{i \in L} X_{ik} = \sum_{j \in L} X_{kj} \hspace{4pt} \forall t \in T \hspace{4pt} \forall k \in L$ 
		\item $X_{ij}^t \leq C_{lineIJ}W_{ij}^t \hspace{4pt} \forall t \in T \hspace{4pt} \forall i,j \in P$
		\item $\sum_{i \in L} C_{RepairTimeI} F_{i}^t + \sum_{i \in L} \sum_{j<i \in L}  K_{ij}^t C_{TravelIJ} C_{broken} \leq 8 \hspace{4pt} \forall t \in T \hspace{4pt}$
		\item $\sum_{j \in L} K_{0j}^t \geq 1$
		\item A subtour elimination constraint, though I'm not sure what the best way to set this up is 
	\end{itemize}
	\section{Comments}
	\begin{itemize}
		\item The first constraint system is not linear, and I'm worried about it causing a mess of runtime.
		\item Does the lack of multiplication by functionality in the second constraint system mean that we could choose X values that are non-zero for a broken node?
	\end{itemize}
\end{document}