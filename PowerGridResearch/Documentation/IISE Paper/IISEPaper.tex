\documentclass[10pt]{article}
\usepackage[dvips]{graphics}
\pagestyle{plain}
\textheight 9in
\textwidth  6.5in
%\footheight 0.0cm
\topmargin  0pt
\headheight 0in
\headsep 0in
\leftmargin 0.0cm
\oddsidemargin  0cm
\def\baselinestretch{1.0}
\parindent 0.0 cm
\parskip   0.2 cm

\newcounter{step}
\newtheorem{theorem}{Theorem}
\newtheorem{proposition}{Proposition}
\newtheorem{corollary}{Corollary}
\newtheorem{definition}{Definition}

\renewcommand{\floatpagefraction}{0.90}
\renewcommand{\topfraction}{0.90}
\renewcommand{\bottomfraction}{0.90}
\renewcommand{\textfraction}{0.10}

\def\thesection{\large{\arabic{section}.}}
\def\thesubsection{\normalsize{\thesection\arabic{subsection}}}
\def\thesubsubsection{\normalsize{\thesection\arabic{subsection}.\arabic{subsubsection}}}


\begin{document}

\newpage
\thispagestyle{empty}
\begin{center}
{\Large \textbf{ Working Title: Multi-Layer Infrastructure Repairs in a Post-Disaster Context}} \\
\vspace*{0.2cm}
{\large \textbf{Brian French and Erhan Kutanoglu}} \\
{\large \textbf{Operations Research and Industrial Engineering Program}} \\
{\large \textbf{The University of Texas at Austin}} \\
{\large \textbf{Austin, TX 78712, USA}} \\
{\large \textbf{BCFrench@utexas.edu, erhank@mail.utexas.edu}} \\
\end{center}

\thispagestyle{empty}
\begin{center}
{\large \bf Abstract}
\end{center}

\vspace*{-12pt}


{\large \bf Keywords}\\


\vspace*{-12pt}
\section{{\large Introduction}}
\label{sec:ic:intro}
\vspace*{-12pt}

Hurricanes provide a growing concern in the operation of power grids in coastal areas. This paper was undertaken to address the gap in the literature where previous efforts don't consider how network layers depend on each other. To affect  repairs on a damaged power grid element, the element must be accessible to the crew attempting to repair it, which implies that the road grid is a part of the repair efforts. In the process of a hurricane, the road grid will sustain substantial damage from flooding and debris on the road surface, which necessitates road grid repairs as well. To handle the issues of repairing power grids efficiently, consideration for repairs of both aspects should be considered jointly. Previous literature 

\subsection{\large Existing Literature}
\vspace*{-12pt}
Damage from hurricanes on infrastructure elements is well studied. All of \cite{WinklerEA2010}, \cite{ScherbEA2015}, and \cite{GuikemaEA2010} have looked at the damage to power grids in varying capacities and methodologies. Damage to the road network is studied more in the context of repairing damage. \cite{AksuEA2014} addresses concerns of accessibility to locations in the wake of road damage. \cite{DuqueEA2016} focuses their work on the ability to move disaster supplies around.

In the context of power grid repair, it comes from two major areas: Network interdiction and work similar to this paper. \cite{SalmeronEA2010} is a paper emblematic of work on interdiction and provide the basis for the linear programming formulation of DC power flow used in this paper. \cite{Wood2011} provides further literature review of the interdiction problem. \cite{NPSMasters} addresses a problem similar to this work, though without addressing travel times at all. \cite{ArabEA2015} and \cite{MousavizadehEA2018} both address repairs to power grids in the context of resilience. Most similar to this paper is \cite{BentEA2011} in that they consider DC power flow and repair with travel times jointly. The key distinction is that they presume that the road operates under nominal conditions rather than including repair of the road grid into the problem.

\section{\large{Model}}
\vspace*{-12pt}
\subsection{Overview}
The problem to be addressed is how to handle interactions between road and power networks during disaster response. We choose to do this by solving the problems independently to keep runtime tractable and then testing a variety of post processing methods to handle the interactions. This should allow us to capture which interactions are worth the effort to address and which can be ignored in the name of preserving runtime or simplifying analysis. 
\subsection{Assumptions}
\vspace*{-12pt}
While considering coordinated repair using the mixed-integer programs below, we assume complete information about damage to both networks, complete information about repair times, and no variation in repair times. In the course of modeling the problem, we assume that a DC-flow model of power grid operation is close enough to real network behavior to draw useful insights \cite{QiEA2012}. We also assume in the power grid repair model that a minimum spanning tree can approximate routing elements NEEDS CITATION as the minimum spanning tree provides a bound on the traveling salesman problem and therefore routing problems.
\subsection{Road Model}
\vspace*{-12pt}
To model the road repair aspect of the problem, we choose to treat it as a problem of routing a crew tasked with clearing debris and flooding from roads. In doing this, damaged roads aren't rendered unable to be transited, but their lengths are merely increased to account for the difficulty traversing them while clearing debris, minor flooding, and the like. This problem is handled by solving a variation of the prize collecting rural postman problem at each time step. The discrete time here is done to match up with both standard shift lengths and the discrete time requirement of the DC power flow based model for the power grid.

Defining Parameters and Sets to begin:
\begin{displaymath}
$$
$\begin{array}{ll}
T & \mbox{the set of time periods (shifts) over the time horizon}\\
N & \mbox{the set of nodes in the graph}\\
C_{ij} & \mbox{ measure of the value of the road from i to j}\\
L_{ij} & \mbox{is the length of the road between nodes $i$ and $j$ when everything is working as normal}\\
R_{ij} & \mbox{time to repair the road between $i$ and $j$}\\
I_{ij} & \mbox{initial condition of the road between $i$ and $j$}\\
\end{array}$
$$
\end{displaymath}

We then elect to minimize the length of non functioning road over the repair horizon. Without loss of generality the objective can be replaced with the repair priority weights for an agency of choice. 
\begin{equation}
	\min \sum_{t \in T}  \sum_{i,j \in N} C_{ij}*(1-X_{ij}^t) 
\end{equation}
	
	
	\mbox{Subject To:}
	
	\begin{eqnarray}
	\sum_{i,j \in N} S_{ij}^t K_{ij}^t \leq 8, \hspace{6pt} \forall t\in T \\
	S_{ij}^t = \max \{L_{ij}, (1-X_{ij}^t)R_{ij} \}, \hspace{6pt} \forall t\in T \hspace{4pt} \forall i,j \in N\\
	\sum_{j \in N} K_{ij}^t - \sum_{j \in N} K_{ji}^t = 0, \hspace{6pt} \forall t\in T \hspace{4pt} \forall i \in N\\
	X_{ij}^t \le \sum_{t'=0}^{t-1} K_{ij}^{t'} + I_{ij}, \hspace{6pt} \forall t\in T \hspace{4pt} \forall i,j \in N\\
	\sum_{i,j \in S; i\neq j} X_{ij}^t \leq |S|-1 \hspace{6pt} \forall S \subset N; \hspace{2pt} S \neq \emptyset
	\end{eqnarray}
Constraint (2) is a shift scheduling constraint capping the length of each tour. Constraint (3) is a linearizable representation of the length of a road conditional on its repair status. Constraints (4) and (6) are standard routing constraints for subtour elimination and path connectivity. Constraint(5) defines the working of damaged roads.


\subsection{Power Model}
\vspace*{-12pt}
To model the power repair aspect of the problem, we treat it primarily as a scheduling problem. This repair schedule is subject to travel time and shift length constraints, which complicate things as well as an embedded DC power-flow model to evaluate each shift. Traditional routing problems are computationally intractable with commercial solvers on this scale, so to keep runtime in check, a spanning tree was used instead as it provides a lower bound on routing and therefore captures most of the tradeoffs \$CITATION NEEDED. NEED TO TALK ABOUT DISCRETE TIME ASSUMPTION?

We begin modeling the power half of the problem by defining the relevant parameters, sets, and design variables:

\small
\begin{displaymath}
$$
$\begin{array}{llll}
	 N & \mbox{set of nodes, indexed by $i$} & X_{e}^{t} & \mbox{flow on line $e$ at time $t$}\\
	 E & \mbox{set of power lines, indexed by $e$} & G_{k}^t & \mbox{production from generator $k$ at time $t$}\\
	 R & \mbox{the set of road segments} & V_i^t & \mbox{indicator for node $i$ functioning at time $t$}\\
	 T & \mbox{the planning horizon, indexed by t} & W_{e}^t & \mbox{indicator for line $e$ functioning at time $t$} \\
	 O(i) & \mbox{set of lines with origin i} & S_{e}^t & \mbox{indicator for line $e$ serviced at time $t$}\\
	 D(i) & \mbox{set of lines with destination i} & F_i^t & \mbox{indicator for node $i$ serviced at time $t$}\\
	 o(e) & \mbox{origin node of line e} & \theta_i^t & \mbox{hase angle for the power flow at $i$ in time $t$}\\
	 d(e) & \mbox{destination node of line e} & MST_t & \mbox{length of the tree used for "routing" at $t$} \\
	 \underline{L_e},\overline{L_e} & \mbox{capacity lower and upper bounds for the power line $e$}& Z_{ij}^t & \mbox{indicator for $i$ to $j$ in the spanning tree at $t$}\\
	 \Delta_{i} & \mbox{time to repair node $i$} \\
	 \delta_{e} & \mbox{time to repair line $e$}\\
	  C_{SP(i)} & \mbox{length of the shortest path to node $i$ from the central depot}\\
	  D_i & \mbox{power demand at location $i$ in the pre-disaster steady state}\\
	  P_k & \mbox{maximum power generation for generator $k$}\\
	  B_e&  \mbox{line susceptance for power line $e$}\\
	 I_e, I_i & \mbox{initial condition of line $e$ and node $i$, respectively}
\end{array}$
$$
\end{displaymath}
\normalsize
The model is then as follows
\begin{equation}
\min \sum_{i \in N} \sum_{t \in T} (1-W_i^t)D_i
\end{equation}
subject to:
\begin{eqnarray}
 X_e^t = B_e * (\theta_{o(e)}^t - \theta_{d(e)}^t), \hspace{5pt} \forall t \in T, \hspace{4pt} \forall e \in E\\
 G_i^t - \sum_{l \in O(i)} X_l^t + \sum_{l \in D(i)} X_l^t = D_i, \hspace{4pt} \forall t \in T, \hspace{4pt} \forall i \in N\\
 G_k^t \leq P_{k} V_{k}^t, \hspace{4pt} \forall t \in T, \hspace{4pt} \forall k \in N\\
 \underline{L_e}W_{e}^t \leq X_{e}^t \leq \overline{L_e}W_{e}^t, \hspace{4pt} \forall t \in T, \hspace{4pt} \forall e \in E\\
 \underline{L_e}V_{o(e)}^t \leq X_{e}^t \leq \overline{L_e}V_{o(e)}^t, \hspace{4pt} \forall t \in T, \hspace{4pt} \forall e \in E\\
 \underline{L_e}V_{d(e)}^t \leq X_{e}^t \leq \overline{L_e}V_{d(e)}^t, \hspace{4pt} \forall t \in T, \hspace{4pt} \forall e \in E\\
 MST^t = \sum_{i \in N} \sum_{j \in N} SP_{ij}^t*Z_{ij}^{t} C_{speed}\hspace{4pt} \forall t \in T\\
 \sum_{i \in N} \sum_{j \in N} Z_{ij}^{t} = \sum_{i \in N} F_i^t + \sum_{e \in E} S_e^t - \sum_{i \in N} F_i^t \sum_{O(i)} S_e^t - \sum_{i \in N} F_i^t \sum_{D(i)} S_e^t \hspace{4pt} \forall t \in T\\
 \sum_{i,j \in S} Z_{ij}^t \leq |S|-1 \hspace{4pt} S\subset N\\
 \sum_{j \in N} Z_{ij}^t \leq F_i^t + \sum_{e \in O(i) \cup D(i)} S_{e}^t \hspace{4pt} \forall t \in T\hspace{4pt} \forall i \in N \\
 \sum_{e \in E} \delta_{e}S_e^t + \sum_{i \in N}\Delta_{i}F_i^t + MST_t <=8\\
 V_i^t \leq \sum_{t'=0}^{t-1} F_i^{t'}+I_i, \hspace{4pt} \forall i \in N\\
 W_{e}^t \leq \sum_{t'=0}^{t-1} S_{e}^{t'}+I_e, \hspace{4pt} \forall e \in E
 \end{eqnarray}
 Constraints (2)-(7) define the flow on a damaged power grid with load being either switched on or switched off at a demand node. This is done to account for the last mile distribution lines being damaged at first making partial load shedding unrealistic.Constraints (8)-(11) handle construction of a minimum spanning tree between nodes being repaired during a shift to approximate the routing cost. We assume that a power line can be fixed starting from either of it's endpoints. Constraint (12) defines shift scheduling, and constraints (13) and (14) handle constraining operation to elements that are working.
\section{\large{Results}}
\vspace*{-12pt}
\subsection{Methods}
 \vspace*{-12pt}
 
 To demonstrate both validity and utility of the model, we generate two test cases, one based on the IEEE 30 bus network and a second based on the IEEE 57 bus network. Both power grids were overlaid with a Watts-Strogatz graph to provide an simulation of a road network for the sake of the routing aspects of the problem. \$CITE-HERE. To generate a damage scenario, a subset of geographically co-located nodes, power lines, and road network edges were marked as damaged to have scenarios to solve.
 
 The scenarios were each solved under the following frameworks:
 \begin{itemize}
 	\item Solve the road repair problem and use the solution to generate a time varying shortest path matrix for use when solving the power repair problem
 	\item Solve the power repair problem assuming statically damaged roads
 	\item Solve the power repair problem assuming nominal road conditions and add a one-shift delay to account for the road repairs being done before the roads are needed
 	\item Solve the power repair problem without travel time and using the generated schedule, post process it into a feasible repair schedule.
 \end{itemize}

In doing this, the problem is solved to a lower bound via just scheduling without travel time, status quo solution via post processing, and several variations of improved solution methods.
%%frameworks?
%%lower bound with schedule
%%upper bound with non-interacting road damage
%%iterative
\subsection{IEEE 30 Bus Case}

The IEEE 30 bus netowkr is a standard power grid that approximately resembles the power grid in southern Illinois in the 1960s \$CITE THIS. More importantly, it's been a well understood standard test grid since it's inception.
\subsection{IEEE 57 Bus Case}

\section{\large{Conclusions}}
\label{sec:issues}
\vspace*{-12pt}

From this, we can draw the conclusion that increasing levels of model complexity representing an improved fidelity to fully coordinated operations yields lower levels of demand shed.  
\bibliographystyle{plain}
\bibliography{sources}

\end{document}
