\documentclass[t, pdftex]{beamer}  
%Use Cockrell School Theme.  Optional department name.  Must %ecape, i.e. use 
%backslash, to preserve spaces.  The default is ``Cockrell School of Engineering''
\usetheme[]{cockrell}                 
%\usetheme[dept=Aerospace\ Engineering\ and\ Engineering\ Mechanics]{cockrell}                 
%\usetheme[dept=Biomedical\ Engineering]{cockrell}                 
%\usetheme[dept=Chemical\ Engineering]{cockrell}                 
%\usetheme[dept=Civil,\ Architectural\ and\ Environmental\ Engineering]{cockrell}                 
%\usetheme[dept=Electrical\ and\ Computer\ Engineering]{cockrell}                 
%\usetheme[dept=Mechanical\ Engineering]{cockrell}                 
%\usetheme[dept=Materials\ Science\ and\ Engineering]{cockrell}                 
%\usetheme[dept=Petroleum\ and\ Geosystems\ Engineering]{cockrell}                 

% Add preamble packages here
%\usepackage{etex}
%\usepackage[bigfiles]{media9}
%\graphicspath{{./figs/}}

%Enable cancelto in math
\usepackage{amsmath}
\usepackage[fleqn]{mathtools}
\usepackage{amssymb}
\usepackage{cancel}
\usepackage{enumitem}
\renewcommand{\CancelColor}{\color{utorange}}

%Add bibliography file location for citiation
\bibliography{}


\title{Joint Post-Disaster Repair of Power and Road Networks}
\subtitle{}
\author{Erhan Kutanoglu and Brian French}
\institute{UT Austin}
\date{\today}
\begin{document}
	\maketitle
	%Creates title frame from title, subtitle, author, institute, and date above
	
	%Supports table of contents
	\frame{\frametitle{Outline}\tableofcontents}
	
	%Section commands will define what's shown in TOC
	\section{Motivating the Problem}
	\frame{\frametitle{Background}
		\begin{itemize}
			\item -Hurricanes cause significant damage to multiple infrastructure layers
			\item -Harvey took almost 11,000 MW of production capacity offline due to wind and flooding
			\item -Flooding rendered large swathes of the road system impossible to traverse, complicating repair efforts
			
		\end{itemize}
		}
	\frame{\frametitle{Existing Literature}
	\begin{itemize}
		\item -Power Grid Resilience: Mousavizadeh et al 2018
		\item -Repair Planning: Ang and Salmeron 2006
		\item -Resource prepositioning: Ali Arab et al 2015
		\item -Supply delivery: Holguin-Veras et al 2012
		\item -Power repair with routing: Bent, Van Hentenryck, and Coffrin 2011
		\item -Of note is no significant literature addressing both road and power aspects at the same time.
	\end{itemize}
}
	\frame{\frametitle{Power Grid Overview}
		\begin{itemize}
			\item -Transmission side
			\begin{itemize}
				\item -Consists of generators, high voltage lines (345kV through 69kV), and substations.
				\item -main emphasis of this talk
				 
			\end{itemize}
			\item -Distribution side
			\begin{itemize}
			\item -Consists of 14kV and low-voltage distribution lines and corresponding step-down transformers
			\item -Can be abstracted away since it always needs repairs post storm
		\end{itemize}
		\end{itemize}
		}
	\section{Pure Power Grid Model}
\frame{
	\frametitle{Power Model: Sets and Indices}
	\begin{itemize}
		\item $N$ is the set of nodes, indexed by $i$
		\item $E$ is the set of power lines, indexed by $e$
		\item $R$ is the set of road segments
		\item $T$ is the planning horizon, indexed by $t$
		\item $O(i)$ is the set of lines with origin $i$
		\item $D(i)$ is the set of lines with destination $i$
		\item $o(e)$ is the origin node of line $e$
		\item $d(e)$ is the destination node of line $e$
	\end{itemize}
	
}
\frame{
	\frametitle{Power Model: Variables}
Variables:
\begin{itemize}
	\item $X_{e}^{t}$ is the flow on line $e$ at time $t$
	\item $G_{k}^t$ is the production from generator $k$ at time $t$
	\item $V_i^t$ is 1 if node $i$ is functioning at time $t$
	\item $W_{e}^t$ is 1 if line $e$ is functioning at time $t$
	\item $S_{e}^t$ is 1 if line $e$ is serviced at time $t$
	\item $F_i^t$ is 1 if node $i$ is serviced at time $t$ 
	\item $\theta_i^t$ is the phase angle for the power flow at $i$ in time $t$
	\item $MST_t$ is the length of the tree used for "routing" at time $t$
	\item $Z_{ij}^t$ is 1 if the shortest path link from $i$ to $j$ is an element in the tree at time $t$	
\end{itemize}
}
\frame{
\frametitle{Power Model: Parameters and Data}
\begin{itemize}
\item $\underline{L_e}$ and $\overline{L_e}$ is the capacity lower and upper bounds for the power line $e$
\item $\Delta_{i}$ is the time to repair node $i$ and $\delta_{e}$ is the time to repair line $e$
\item $C_{SP(i)}$ is the length of the shortest path to node $i$ from the central depot
\item $D_i$ is the power demand at location $i$ in the pre-disaster steady state
\item $P_k$ is the maximum power generation for generator $k$
\item $B_e$ is the line susceptance (imaginary part of admittance, also inverse of resistance) for power line $e$
\item $I_e, I_i$ is the initial condition of line $e$ and node $i$, respectively.
\end{itemize}
}
\frame{
\frametitle{The Model}
Objective:
$$ \min \sum_{i \in N} \sum_{t \in T} (1-W_i^t)D_i $$
Subject to:
\begin{enumerate}[label=(\arabic*), leftmargin=*, itemsep=0.4ex, before={\everymath{\displaystyle}}]

\item $ X_e^t = B_e * (\theta_{o(e)}^t - \theta_{d(e)}^t), \hspace{5pt} \forall t \in T, \hspace{4pt} \forall e \in E$
\item $ G_i^t - \sum_{l \in O(i)} X_l^t + \sum_{l \in D(i)} X_l^t = D_i, \hspace{4pt} \forall t \in T, \hspace{4pt} \forall i \in N$
\item $G_k^t \leq P_{k} V_{k}^t, \hspace{4pt} \forall t \in T, \hspace{4pt} \forall k \in N$
\item $\underline{L_e}W_{e}^t \leq X_{e}^t \leq \overline{L_e}W_{e}^t, \hspace{4pt} \forall t \in T, \hspace{4pt} \forall e \in E$
\item $\underline{L_e}V_{o(e)}^t \leq X_{e}^t \leq \overline{L_e}V_{o(e)}^t, \hspace{4pt} \forall t \in T, \hspace{4pt} \forall e \in E$
\item $\underline{L_e}V_{d(e)}^t \leq X_{e}^t \leq \overline{L_e}V_{d(e)}^t, \hspace{4pt} \forall t \in T, \hspace{4pt} \forall e \in E$
\end{enumerate}
These 6 constraints define the mathematical rules for operation of a damaged DC-approximated power grid.
}
\frame{
	\frametitle{The Model Continued}
	Subject to:
	\begin{enumerate}[label=(\arabic*), leftmargin=*, itemsep=0.4ex, before={\everymath{\displaystyle}}]
		
	\item $MST^t = \sum_{i \in N} \sum_{j \in N} SP_{ij}^t*Z_{ij}^{t} C_{speed}\hspace{4pt} \forall t \in T $
	\item $\sum_{i \in N} \sum_{j \in N} Z_{ij}^{t} = \sum_{i \in N} F_i^t + \sum_{e \in E} S_e^t - \sum_{i \in N} F_i^t \sum_{O(i)} S_e^t - \sum_{i \in N} F_i^t \sum_{D(i)} S_e^t \hspace{4pt} \forall t \in T$
	\item $\sum_{i,j \in S} Z_{ij}^t \leq |S|-1 \hspace{4pt} S\subset N$
	\item $ \sum_{j \in N} Z_{ij}^t \leq F_i^t + \sum_{e \in O(i) \cup D(i)} S_{e}^t \hspace{4pt} \forall t \in T\hspace{4pt} \forall i \in N $
	
	These 4 constraints lay out how to construct a spanning tree among repaired elements to handle "routing"
	\end{enumerate}

}
\frame{\frametitle{The Model Continued}
	Subject to:
\begin{enumerate}[label=(\arabic*), leftmargin=*, itemsep=0.4ex, before={\everymath{\displaystyle}}]
	\item $ \sum_{e \in E} \delta_{e}S_e^t + \sum_{i \in N}\Delta_{i}F_i^t + MST_t <=8$

\item $V_i^t \leq \sum_{t'=0}^{t-1} F_i^{t'}+I_i, \hspace{4pt} \forall i \in N$ 
\item $W_{e}^t \leq \sum_{t'=0}^{t-1} S_{e}^{t'}+I_e, \hspace{4pt} \forall e \in E $
	\end{enumerate}

These final 3 constraints govern shift scheduling and definition of functionality of grid elements
}
\section{Pure Road Model}
	\frame{\frametitle{Pure Road Grid Model}
	Sets:
	\begin{itemize}
		\item $T$ is the set of time periods (shifts) over the time horizon
		\item $N$ is the set of nodes in the graph
	\end{itemize}
	
	Parameters:
	\begin{itemize}
		\item $C_{ij}$ is a measure of the value of the road to relief supply delivery efforts
		\item $L_{ij}$ is the length of the road between nodes $i$ and $j$ when everything is working as normal
		\item $R_{ij}$ is the time to repair the road between $i$ and $j$
		\item $I_{ij}$ is the initial condition of the road between $i$ and $j$
	\end{itemize}
	
	Variables:
	\begin{itemize}
		\item $X_{ij}^t$ is 1 if the road between nodes $i$ and $j$ is working at time $t$
		\item $S_{ij}^t$ is the length of the road between $i$ and $j$ at time $t$. 
		\item $K_{ij}^t$ is 1 if $j$ follows $i$ in the tour at time $t$, and 0 otherwise.	
	\end{itemize}
}
\frame{\frametitle{Pure Road Grid Model Continued}
	$$	\min \sum_{t \in T} t \sum_{i,j \in N} C_{ij}*(1-X_{ij}^t) $$

subject to:
\begin{enumerate}[label=(\arabic*), leftmargin=*, itemsep=0.4ex, before={\everymath{\displaystyle}}]%
	\item $\sum_{i,j \in N} S_{ij}^t K_{ij}^t \leq 8, \hspace{6pt} \forall t\in T$
	\item $S_{ij}^t = \max \{L_{ij}, (1-X_{ij}^t)R_{ij} \}, \hspace{6pt} \forall t\in T \hspace{4pt} \forall i,j \in N$
	\item $\sum_{j \in N} K_{ij}^t - \sum_{j \in N} K_{ji}^t = 0, \hspace{6pt} \forall t\in T \hspace{4pt} \forall i \in N$
	\item $X_{ij}^t \le \sum_{t'=0}^{t-1} K_{ij}^{t'} + I_{ij}, \hspace{6pt} \forall t\in T \hspace{4pt} \forall i,j \in N$
	\item $\sum_{i,j \in S; i\neq j} X_{ij}^t \leq |S|-1 \hspace{6pt} \forall S \subset N; \hspace{2pt} S \neq \emptyset$
\end{enumerate}
}
\section{Scenario}
	\frame{\frametitle{Constructed Grid for Example Results}
	\begin{itemize}
		\item -We overlay IEEE 30 bus based power grid onto a grid to assign distances between elements
		\item -Nodes are chosen to be connected directly by roads that are modeled as a Watts-Strogatz graph
		\item -a representation of the system is shown below with power in green and roads in red.
	\end{itemize}
	\includegraphics[scale=0.5]{"Mock Network".png}
}
	\frame{\frametitle{Scenario}
	To simulate damage to a power network, we mark the following as damaged:
	\begin{itemize}
		\item Power Grid Nodes: 4,7,18,23,24,27
		\item Power Grid Edges: (1,4),(4,6),(7,27),(24,25),(11,15),(1,3),(18,19),(9,22),(9,19)
	\end{itemize}
	We then randomly damage half the road edges in a way that represents debris and minor flooding giving them a repair time 8 times the normal transit time.
	\begin{itemize}
		\item Road Edges: (0,3),(1,2),(1,26),(4,5),(5,13),(9,10),(9,13),(10,11),
		(10,12),(13,15),(14,16),(14,28),(16,17),(17,22),(18,21),(19,21),
		(19,23),(22,24),(22,25),(22,26),(24,27),(25,27)
	\end{itemize}
}
\section{Preliminary Results}


	\frame{\frametitle{Results for Road}
			\begin{itemize}
			\item -Assumption is that the road utility wants to maximize the length of working roads
			\item -All tours have to start and end at node 13, which was chosen arbitrarily
		\end{itemize}
	{\centering
		\begin{tabular}{|c|c|}
			\hline
			Shift 1 & (21,18),(22,17),(25,22),(27,25)\\
			\hline
			Shift 2 & (9,13),(13,9),(24,22),(27,24) \\
			\hline
			Shift 3 & (17,22),(18,21),(22,25),(25,27) \\
			\hline
			Shift 4 & (10,9),(19,21),(22,24),(23,19),(24,27) \\
			\hline
			Shift 5 & (5,13),(13,5) \\
			\hline
			Shift 6 & Nothing fixed due to end of horizon effects
		\end{tabular}\par
	}
}
\frame{\frametitle{Results for Power}
	\begin{itemize}
		\item -Objective 469.18
		\item -Assumption is that all roads are repaired before they're used bt the power repair crews need them
		\item -Repair Schedule below
	\end{itemize}
    
	{\centering
	\begin{tabular}{|c|c|}
		\hline
		Shift 1 & Bus 4 and Lines (11,15) \\
		\hline
		Shift 2 & Lines (1,4) and (9,19)\\
		\hline
		Shift 3 & Bus 7 and Lines (4,6)\\
		\hline
		Shift 4 & Bus 23 and Line (24,25) \\
		\hline
		Shift 5 & Bus 24 Line (9/22) \\
		\hline
		Shift 6 &  \\
		\hline
	\end{tabular}\par
}
}
\frame{\frametitle{Visualizations}
	\begin{figure}
\includegraphics[scale=0.35]{"LoadShedPower".png}

\caption{A comparison of the amount of load shed when road integrity is considered}

\end{figure}
}
\frame{
	\begin{figure}
	\figurename{Road Tour}
\includegraphics[scale = .7]{"T0 Tour".png}
\caption{The tour to clear/repair roads at the first time step}
\end{figure}
}
\frame{\frametitle{Conclusions from non-iterative approach}
\begin{itemize}
	\item -solutions exhibit geographical clustering
	\subitem -effect not seen in some previous work
	\item -fast repairs function of small network size used in example
	\item -some roads are not repaired because it would require routing along multiple damaged roads and end up violating shift length constraint
	\item -Power repairs will depend on what roads work, need to think of them together
\end{itemize}
}

\section{Iterative Solutions}
\frame{\frametitle{Iterative Road First Results}
	
	Objective Value 552.107
	\newline
	
	{\centering
		\begin{tabular}{|c|c|}
			\hline
			Shift 1 & Bus 7 and Line (4/6) \\
			\hline
			Shift 2 &  Lines (1,3),(1,4) \\
			\hline
			Shift 3 & Bus 4 and Line (18,19)\\
			\hline
			Shift 4 &  Bus 18 and Line (7,27) \\
			\hline
			Shift 5 & Bus 23 and Line (24,25)\\
			\hline
			Shift 6 &  \\
			\hline
		\end{tabular}\par

}
}
	\frame{\frametitle{Conclusions}
	\begin{itemize}
		\item -inclusion of road repair meaningfully changes repair schedule
		\item -this suggests this as a useful direction of study
		\item -non-contrived example should be done to better showcase the interactions on a real geography
		\subitem -ACTIVS2000 simulated network and Texas topology provide for this
	\end{itemize}
}
		
\end{document}