\documentclass[t, pdftex]{beamer}  
%Use Cockrell School Theme.  Optional department name.  Must %ecape, i.e. use 
%backslash, to preserve spaces.  The default is ``Cockrell School of Engineering''
\usetheme[]{cockrell}                 
%\usetheme[dept=Aerospace\ Engineering\ and\ Engineering\ Mechanics]{cockrell}                 
%\usetheme[dept=Biomedical\ Engineering]{cockrell}                 
%\usetheme[dept=Chemical\ Engineering]{cockrell}                 
%\usetheme[dept=Civil,\ Architectural\ and\ Environmental\ Engineering]{cockrell}                 
%\usetheme[dept=Electrical\ and\ Computer\ Engineering]{cockrell}                 
%\usetheme[dept=Mechanical\ Engineering]{cockrell}                 
%\usetheme[dept=Materials\ Science\ and\ Engineering]{cockrell}                 
%\usetheme[dept=Petroleum\ and\ Geosystems\ Engineering]{cockrell}                 

% Add preamble packages here
%\usepackage{etex}
%\usepackage[bigfiles]{media9}
%\graphicspath{{./figs/}}

%Enable cancelto in math
\usepackage{amsmath}
\usepackage[fleqn]{mathtools}
\usepackage{amssymb}
\usepackage{cancel}
\usepackage{enumitem}
\renewcommand{\CancelColor}{\color{utorange}}

%Add bibliography file location for citiation
\bibliography{}


\title{Joint Post-Disaster Repair of Power and Road Networks}
\subtitle{}
\author{Erhan Kutanoglu and Brian French}
\institute{UT Austin}
\date{\today}
\begin{document}
	\maketitle
	%Creates title frame from title, subtitle, author, institute, and date above
	
	%Supports table of contents
	\frame{\frametitle{Outline}\tableofcontents}
	
	%Section commands will define what's shown in TOC
	\section{Motivating the Problem}
	\frame{\frametitle{Background}
		\begin{itemize}
			\item Hurricanes cause significant damage to multiple infrastructure layers
			\item Harvey took almost 11,000 MW of production capacity offline due to wind and flooding
			\item Flooding rendered large swathes of the road system impossible to traverse, complicating repair efforts
			
		\end{itemize}
		}
	\frame{\frametitle{Power Grid Overview}
		\begin{itemize}
			\item Transmission side
			\begin{itemize}
				\item Consists of generators, high voltage lines (345kV through 69kV), and substations.
				\item main emphasis of this talk
				 
			\end{itemize}
			\item Distribution side
			\begin{itemize}
			\item Consists of 14kV and low-voltage distribution lines and corresponding step-down transformers
			\item Can be abstracted away a bit due to being a better understood problem.
		\end{itemize}
		\end{itemize}
		}
	\frame{\frametitle{Pure Power Grid Model}
	
}
	\frame{\frametitle{Pure Road Grid Model}
	Sets:
	\begin{itemize}
		\item $T$ is the set of time periods (shifts) over the time horizon
		\item $N$ is the set of nodes in the graph
	\end{itemize}
	
	Parameters:
	\begin{itemize}
		\item $C_{ij}$ is a measure of the value of the road to relief supply delivery efforts
		\item $L_{ij}$ is the length of the road between nodes $i$ and $j$ when everything is working as normal
		\item $R_{ij}$ is the time to repair the road between $i$ and $j$
		\item $I_{ij}$ is the initial condition of the road between $i$ and $j$
	\end{itemize}
	
	Variables:
	\begin{itemize}
		\item $X_{ij}^t$ is 1 if the road between nodes $i$ and $j$ is working at time $t$
		\item $S_{ij}^t$ is the length of the road between $i$ and $j$ at time $t$. 
		\item $K_{ij}^t$ is 1 if $j$ follows $i$ in the tour at time $t$, and 0 otherwise.	
	\end{itemize}
}
\frame{\frametitle{Pure Road Grid Model Continued}
	$$	\min \sum_{t \in T} t \sum_{i,j \in N} C_{ij}*(1-X_{ij}^t) $$

subject to:
\begin{enumerate}[label=(\arabic*), leftmargin=*, itemsep=0.4ex, before={\everymath{\displaystyle}}]%
	\item $\sum_{i,j \in N} S_{ij}^t K_{ij}^t \leq 8, \hspace{6pt} \forall t\in T$
	\item $S_{ij}^t = \max \{L_{ij}, (1-X_{ij}^t)R_{ij} \}, \hspace{6pt} \forall t\in T \hspace{4pt} \forall i,j \in N$
	\item $\sum_{j \in N} K_{ij}^t - \sum_{j \in N} K_{ji}^t = 0, \hspace{6pt} \forall t\in T \hspace{4pt} \forall i \in N$
	\item $X_{ij}^t \le \sum_{t'=0}^{t-1} K_{ij}^{t'} + I_{ij}, \hspace{6pt} \forall t\in T \hspace{4pt} \forall i,j \in N$
	\item $\sum_{i,j \in S; i\neq j} X_{ij}^t \leq |S|-1 \hspace{6pt} \forall S \subset N; \hspace{2pt} S \neq \emptyset$
\end{enumerate}
}
	\frame{\frametitle{Constructed Grid for Example Results}
	\begin{itemize}
		\item We overlay IEEE 30 bus based power grid onto a grid to assign distances between elements
		\item Nodes are chosen to be connected directly by roads that are modeled as a Watts-Strogatz graph
		\item a representation of the system is shown below with power in green and roads in red.
	\end{itemize}
	\includegraphics[scale=0.5]{"Mock Network".png}
}
	\frame{\frametitle{Scenario}
	To simulate damage to a power network, we mark the following as damaged:
	\begin{itemize}
		\item Power Grid Nodes: 6,18,23,27
		\item Power Grid Edges: (1,4),(1,6),(7,27),(24,25),(11,15)(1,3)(18,19)
		\item Road Edges: (5,13),(9,13),(14,16),(14,28),(18,20),(18,21),(22,26)
	\end{itemize}
}
	\frame{\frametitle{Results}
	
}
	\frame{\frametitle{Conclusions}
	
}
		
\end{document}