\documentclass{article}
\usepackage{amsmath}
\usepackage[fleqn]{mathtools}
\usepackage{amssymb}
\usepackage{amsthm}
\usepackage{enumitem}
\usepackage[backend=bibtex]{biblatex}
\bibliography{Sources} 
\begin{document}
\title{A Work-in-Progress Summary of Power/Road Recovery}
\author{Dr. Erhan Kutanoglu and Brian French}
\maketitle

	\section{Introduction and Existing Literature}
	Recently disaster response and resilience has been the ''hot field" in humanitarian logistics, inventory planning, and power grid electrical engineering. There has been a concentrated effort to study several problems that connect these issues across disciplines. Electrical engineering approaches tend to emphasize changes to the structure of the power grid \cite{PanteliEA2016} \cite{OuyangEA2014} in order to improve resilience. Large parts of the literature base on resilient networks in a power grid context grew out of earlier work on multi-actor attack-defense problems \cite{SalmeronEA2004} \cite{MolyneauxEA2016}. 

	
	Some more recent work has served to be a crossover between operations research style tactical and operational response planning and electrical engineering style strategic planning for the long term. \cite{ArabEA2015} provides a good introduction with the interaction of electrical engineering treatment of power grids with the operations research approach to logistics. While the model is more focused on pre-planning than response, it serves to demonstrate how a joint approach can be very effective. \cite{MousavizadehEA2018} provides a better extension showing the interactions between microgrid formation in the power grid topology and disaster response.
		

	
	Many of these projects miss interactions between the solutions of multiple infrastructure systems such as the road grid being necessary to move supplies for the power grid's repair. This is likely due to disaster response being a problem of many actors each acting on their own share of the problem with little consideration for the bigger picture \cite{VerasEA2012}. A frequent assumption made is that travel times are small enough to be ignorable when looking at disaster response outside of purely goods flow or routing problems. \cite{NPSMasters} solves a similar problem to what we're looking at under that assumption. It lacks any treatment of road networks and solves power grid repair as a scheduling problem. In the wake of a hurricane, roads will frequently sustain damage \cite{Houser2009} and will need service ranging from debris clearance to road rebuilding. Both \cite{AksuEA2014} and \cite{DuqueEA2016} address various aspects of the road repair/clearing problem.
	
	
	\section{Current Model}
	\subsection{Power Grid}
	Summary: Find a set of power grid elements (buses and lines) to repair in each 8 hour shift of work so that the total load shed over time is minimized. This is subject to the operation of a grid using DC-approximation to power flow and the ability to find a path for the repair crew to reach everything. For now inclusion of proper routing would complicate the problem to a degree where even simple cases would be hard to set up, so routing is approximated using shortest paths right now. This provides a maximal bound on the length of the path a repair crew would take. 
	
	\subsection{DC Power Flow/Minimum Spanning Tree "Routing"}	

	\subsubsection{Notation}
Sets and Indices:
	\begin{itemize}
		\item $N$ is the set of nodes, indexed by $i$
		\item $E$ is the set of power lines, indexed by $e$
		\item $R$ is the set of road segments
		\item $T$ is the planning horizon, indexed by $t$
		\item $O(i)$ is the set of lines with origin $i$
		\item $D(i)$ is the set of lines with destination $i$
		\item $o(e)$ is the origin node of line $e$
		\item $d(e)$ is the destination node of line $e$
	\end{itemize}

Parameters:
	\begin{itemize}
		\item $\underline{L_e}$ and $\overline{L_e}$ is the capacity lower and upper bounds for the power line $e$
		\item $\Delta_{i}$ is the time to repair node $i$
		\item $\delta_{e}$ is the time to repair line $e$
		\item $C_{SP(i)}$ is the length of the shortest path to node $i$ from the central depot
		\item $K$ is a coefficient of "broken-ness" representing the average slowdown from debris on the road and minor flooding
		\item $D_i$ is the power demand at location $i$ in the pre-disaster steady state
		\item $P_k$ is the maximum power generation for generator $k$
		\item $B_e$ is the line susceptance (imaginary part of admittance, also inverse of resistance) for power line $e$
		\item $I_e, I_i$ is the initial condition of line $e$ and node $i$, respectively.
	\end{itemize}

Variables:
	\begin{itemize}
		\item $X_{e}^{t}$ is the flow on line $e$ at time $t$
		\item $G_{k}^t$ is the production from generator $k$ at time $t$
		\item $V_i^t$ is 1 if node $i$ is functioning at time $t$
		\item $W_{e}^t$ is 1 if line $e$ is functioning at time $t$
		\item $S_{e}^t$ is 1 if line $e$ is serviced at time $t$
		\item $F_i^t$ is 1 if node $i$ is serviced at time $t$ 
		\item $\theta_i^t$ is the phase angle for the power flow at $i$ in time $t$
		\item $MST_t$ is the length of the tree used for "routing" at time $t$
		\item $Z_{ij}^t$ is 1 if the shortest path link from $i$ to $j$ is an element in the tree at time $t$	
	\end{itemize}
	
	\subsubsection{Model}
	$$	\min \sum_{i \in N} \sum_{t \in T} (1-W_i^t)D_i $$
	subject to:
	\begin{enumerate}[label=(\arabic*), leftmargin=*, itemsep=0.4ex, before={\everymath{\displaystyle}}]%
		
		\item $ X_e^t = B_e * (\theta_{o(e)}^t - \theta_{d(e)}^t), \hspace{5pt} \forall t \in T, \hspace{4pt} \forall e \in E$
		\item $ G_i^t - \sum_{l \in O(i)} X_l^t + \sum_{l \in D(i)} X_l^t = D_i, \hspace{4pt} \forall t \in T, \hspace{4pt} \forall i \in N$
		\item $G_k^t \leq P_{k} V_{k}^t, \hspace{4pt} \forall t \in T, \hspace{4pt} \forall k \in N$
		\item $\underline{L_e}W_{e}^t \leq X_{e}^t \leq \overline{L_e}W_{e}^t, \hspace{4pt} \forall t \in T, \hspace{4pt} \forall e \in E$
		\item $\underline{L_e}V_{o(e)}^t \leq X_{e}^t \leq \overline{L_e}V_{o(e)}^t, \hspace{4pt} \forall t \in T, \hspace{4pt} \forall e \in E$
		\item $\underline{L_e}V_{d(e)}^t \leq X_{e}^t \leq \overline{L_e}V_{d(e)}^t, \hspace{4pt} \forall t \in T, \hspace{4pt} \forall e \in E$
		\item $MST^t = \sum_{i \in N} \sum_{j \in N} SP_{ij}*Z_{ij}^{t} C_{speed}\hspace{4pt} \forall t \in T $
		\item $\sum_{i \in N} \sum_{j \in N} Z_{ij}^{t} = \sum_{i \in N} F_i^t + \sum_{e \in E} S_e^t - \sum_{i \in N} F_i^t \sum_{O(i)} S_e^t - \sum_{i \in N} F_i^t \sum_{D(i)} S_e^t \hspace{4pt} \forall t \in T$
		\item $\sum_{i,j \in S} Z_{ij}^t \leq |S|-1 \hspace{4pt} S\subset N$
		\item $ \sum_{j \in N} Z_{ij}^t \geq F_i^t \hspace{4pt} \forall t \in T\hspace{4pt} \forall i \in N $
		\item $ \sum_{j \in N} Z_{ij}^t \geq \sum_{o(e) = i \cup d(e) = i} S_e^t$
		\item $ \sum_{e \in E} \delta_{e}S_e^t + \sum_{i \in N}\Delta_{i}F_i^t + MST_t <=8$
		
		\item $V_i^t \leq \sum_{t'=0}^{t-1} F_i^{t'}+I_i, \hspace{4pt} \forall i \in N$ 
		\item $W_{e}^t \leq \sum_{t'=0}^{t-1} S_{e}^{t'}+I_e, \hspace{4pt} \forall e \in E $
	\end{enumerate}
	
Explanation of Constraint Systems:
	\begin{itemize}
		\item Constraint (1) defines flows based on line limits and line susceptance as per Salmeron, Ross, and Baldick 2004 \cite{SalmeronEA2004}
		\item Constraint (2) defines node power balance so that inflow has to match outflow at each node.
		\item Constraint (3) constrains power generation to be in the realm of feasible production conditional on the relevant node being operational
		\item Constraints (4)-(6) constrains line flows to be inside line capacity conditional on the relevant elements being operational
		\item Constraint (7) defines the length of a tree in terms of the shortest path between nodes
		\item Constraint (8) is an inclusion/exclusion count for what elements need to be included in the tree. Because we allow for line repair to occur from either end node of the line, there is a chance of repairing both line and node in the same time step, and the constraint to account for this becomes nonlinear. This is linearizable, but it's clearer to express this way.
		\item Constraint (9) is elimination of subtours in the tree. We can reduce these to just elimination of subtours among damaged nodes in a graph that has been fully connected through use of adding links with length equal to the shortest path between two nodes.
		\item Constraints (10) and (11) dictate which elements have to be included in the spanning tree
		\item Constraint (12) is a shift scheduling constraint so that only 8 hours of things can be done each shift
		\item Constraints (13) and (14) limit elements so that they can only be used if they've been fixed.
	\end{itemize}
	
	\subsubsection{Comments}
	\begin{itemize}
		\item This model also assumes DC power flow, which is a much more thorough version of power flow than pipeflow-style models
		\item Note that from constraints (13) and (14), once an element is working, we can chose whether or not it is engaged or turned off to allow load to be shed
	\end{itemize}
	
	\subsubsection{Preliminary Results}
	\begin{figure}
	\includegraphics[scale=0.75]{"Mock Network".png}
	\caption{A representation of a road/power dual network} \label{mocknetwork}
	\end{figure}
	Shown in Figure \ref{mocknetwork} is a representation of IEEE Bus30 (power lines represented in Green) overlaid with a Watts-Newman-Strogatz graph in Red to represent the road grid \cite{NewmanEA2001}. For a contrived scenario to schedule grid repairs, we mark buses 6, 27, 23, 18, and 15 as damaged with a repair time of 5 hours. We mark power lines running between buses (1/4), (4/6), (7,27), (24/25), (11/15), (1/3), and (18/19) as damaged with a repair time of 1 hour. We obtain the following schedule of repairs using the above model:\newline

		{\centering
	\begin{tabular}{|c|c|}
		\hline
		Shift 1 & Bus 6 and Lines (1/4), (11/15) \\
		\hline
		Shift 2 & Bus 23 and Line (24/25) \\
		\hline
		Shift 3 & Bus 18 and Lines (1/3), (4/6)\\
		\hline
		Shift 4 &  \\
		\hline
		Shift 5 &  \\
		\hline
		Shift 6 &  \\
		\hline
	\end{tabular}\par
}
	
	We can see from the above schedule that there is a balance struck between repair of lines and buses to keep the ability to supply demand consistent with production and connection. Worth noting from these preliminary results is that lines (7/27) and (18/19) are not scheduled for repair since it is redundant to the network. In a scenario of hurricane recovery, having knowledge of which lines can be left out of the repair schedule until it is convenient and the bigger problems have been addressed could be a useful perk of the model. 
	  
	\subsection{Road Grid - Basic Routing Repair}
	Summary: Find a tour at each time step that corresponds to less than 8 hours of work to do in a way that minimizes the total cost of damaged roads. Cost is for now just the length of the road, but it's trivial to change the valuation of roads to their utility to a humanitarian response agency or other similar criterion.
	
	\subsubsection{Notation}
Sets:
	\begin{itemize}
		\item $T$ is the set of time periods (shifts) over the time horizon
		\item $N$ is the set of nodes in the graph
	\end{itemize}

Parameters:
	\begin{itemize}
		\item $C_{ij}$ is a measure of the value of the road to relief supply delivery efforts
		\item $L_{ij}$ is the length of the road between nodes $i$ and $j$ when everything is working as normal
		\item $R_{ij}$ is the time to repair the road between $i$ and $j$
		\item $I_{ij}$ is the initial condition of the road between $i$ and $j$
	\end{itemize}
	
Variables:
	\begin{itemize}
		\item $X_{ij}^t$ is 1 if the road between nodes $i$ and $j$ is working at time $t$
		\item $S_{ij}^t$ is the length of the road between $i$ and $j$ at time $t$. 
		\item $K_{ij}^t$ is 1 if $j$ follows $i$ in the tour at time $t$, and 0 otherwise.	
	\end{itemize}
	
	\subsubsection{Model}
	$$	\min \sum_{t \in T} t \sum_{i,j \in N} C_{ij}*(1-X_{ij}^t) $$
	
	subject to:
	\begin{enumerate}[label=(\arabic*), leftmargin=*, itemsep=0.4ex, before={\everymath{\displaystyle}}]%
		\item $\sum_{i,j \in N} S_{ij}^t K_{ij}^t \leq 8, \hspace{6pt} \forall t\in T$
		\item $S_{ij}^t = \max \{L_{ij}, (1-X_{ij}^t)R_{ij} \}, \hspace{6pt} \forall t\in T \hspace{4pt} \forall i,j \in N$
		\item $\sum_{j \in N} K_{ij}^t - \sum_{j \in N} K_{ji}^t = 0, \hspace{6pt} \forall t\in T \hspace{4pt} \forall i \in N$
		\item $X_{ij}^t \le \sum_{t'=0}^{t-1} K_{ij}^{t'} + I_{ij}, \hspace{6pt} \forall t\in T \hspace{4pt} \forall i,j \in N$
		\item $\sum_{i,j \in S; i\neq j} X_{ij}^t \leq |S|-1 \hspace{6pt} \forall S \subset N; \hspace{2pt} S \neq \emptyset$
	\end{enumerate}
	
Explanation of Constraint Systems:
	\begin{itemize}
		\item Constraint (1) is a scheduling constraint so that each tour is less than 8 hours of work
		\item Constraint (2) defines the length of a road to be either the travel length if it is working or the repair cost if it has not been repaired yet
		\item Constraint (3) is path connectivity for the tour
		\item Constraint (4) defines the functionality of each road. While it does not bind to 1, because there is a penalty for not being 1, it will choose 1 if possible
		\item Constraint (5) eliminates subtours to ensure a valid tour
	\end{itemize}

\subsubsection{Preliminary Results}

We mark road grid elements (5/13), (9/13), (14/16), (14/28), (18/20), (18/21), and (22/26) as having sustained minor damage with a time to clear debris equal to 5 times the normal travel time of the road segment. Using the above model, we schedule the following tours:

{\centering
	\begin{tabular}{|c|c|}
		\hline
		Shift 1 & [13,5,4,3,6,7,21,18,23,27,24,22,17,10,13] \\
		\hline
		Shift 2 & [13,12,10,17,22,26,25,27,23,19,20,7,6,3,4,5,13] \\
		\hline
		Shift 3 & [13,9,13] \\
		\hline
		Shift 4 & [13,10,17,22,7,6,3,4,5,13] \\
		\hline
		Shift 5 &  \\
		\hline
		Shift 6 &  \\
		\hline
	\end{tabular}\par
}




\subsection{Iterative Solutions}

\subsubsection{Road First}

When finding optimal repair schedules, both actors solving their problem independently will come to different solutions than when the solutions need to be integrated, and this is a common shortcoming in existing models. The simpler of the two dependent solutions is to solve the road network first. 

We do this by assuming the road network is damaged as it is in the road-repair scenario, but instead of treating the damage as static, we allow links in the road to be traversed in the time step after they've been repaired. The model solves as the model in section 2.2, but the $SP_{ij}$ term in constraint (7) gains a time index and is built from the output of section 2.3's model.

\subsubsection{Power First}

Unlike the road first iterative solution, solving power first doesn't just recycle an earlier model. Given the repairs to be done from the power model without road damage, the optimal route for each shift needs to be computed under existing road damage. For any shift where the route would make the shift cost longer than 8 hours, constraints need to be put in place that repairs must occur for there to exist a route shorter than 8 hours including the repairs for that shift. In the case this generates infeasibilities, lagrangian relaxation should be used to best handle violations of the 8 hour constraint.
	\section{Roadmap}
Currently work in progress
\begin{itemize}
	\item Drawing conclusions from the road first iterative
	\item finishing writing code for the power first iterative and draw conclusions from it
	\item doing a second example with more severe damage on IEEE Bus30
	\item implementing IEEE Bus 118 to show that code scales well
	\item partition out Houston from the ACTIVS2000 grid and map it to real road topology
	\item implement on that grid
\end{itemize}

	\section{Sources}
	\printbibliography
\end{document}