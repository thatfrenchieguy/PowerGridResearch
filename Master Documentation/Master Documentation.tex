\documentclass{article}
\usepackage{amsmath}
\usepackage[fleqn]{mathtools}
\usepackage{amssymb}
\usepackage{amsthm}
\usepackage{enumitem}

\begin{document}
	\section{Introduction}
	\section{Roadmap}
	\begin{itemize}
		\item Exploratory Inventory Location Problem for Newsvendor pre-allocation
		\item contrived grid/contrived geography no-routing power grid repair scheduling with 1 crew
		\item contrived grid/contrived geography perfect information road repair with 1 crew
		\item inclusion of routing into power grid
		\item formulation of real geography and transition to GIS data
		\item transition from pipeflow power to DC grid model
		\item consult electrical engineers RE: power grid analysis
		\item generate real hurricane scenarios
		\item 1-2 response style interactions
		\item 1-2-1 revision style interactions
		\item joint solution
		\item incomplete information in roads
		
	\end{itemize}
	\section{Exploratary ILP}
	\section{Power Grid Models}
	\subsection{NonInteracting/NonRouting}
	\subsection{NonInteracting/Routing}
	\subsubsection{Variable Glossary}
	\begin{itemize}
		\item $C_{FlowIJ}$ is steady state flow on the grid before the hurricane
		\item $C_{lineIJ}$ is the capacity limit for the power line going from IJ
		\item $C_{RepairTimeI}$ is the time to repair node I
		\item $C_{RepairTimeIJ}$ is the time to repair line IJ
		\item $C_{TravelIJ}$ is the travel time between nodes I and J
		\item $C_{broken}$ is a coefficient of "broken-ness" representing the average slowdown from debris on the road and minor flooding
		\item $C_{demandI}$ is the power demand at location I in the pre-disaster steady state
		\item $C_{GeneratorCapacityK}$ is the maximum power generation for generator K
	\end{itemize}
	\subsubsection{Variables}
	\begin{itemize}
		
		\item $X_{ij}^{t}$ is the flow from i to j at time t
		\item $G_{k}^t$ is the production from generator k at time t
		\item $Y_i^t$ is 1 if node i is functioning at time t
		\item $W_{ij}^t$ is 1 if line ij is functioning at time t
		\item $S_{ij}^t$ is 1 if line ij is serviced at time t
		\item $K_{ij}^t$ is 1 if node j follows node i in the tour at time t 
		\item $F_i^t$ is 1 is node i is serviced at time t 
		\item $E_{ij}^t$ is 1 is line ij is serviced at time t	
\end{itemize}
	\subsubsection{Sets}
	\begin{itemize}
		\item L is the set of nodes
		\item P is the set of power lines
		\item R is the set of roads
		\item T is the planning horizon
		\item J is the set of Generators
	\end{itemize}
	\subsubsection{Model}
	$$	Minimize \sum_{i \in L} \sum_{t \in T} C_{FlowIJ}-X_{ij}^t $$
	
	Subject to:
	\begin{enumerate}[label=(\arabic*), leftmargin=*, itemsep=0.4ex, before={\everymath{\displaystyle}}]%
		
		\item $\sum_{i \in L} X_{ik}+G_{k}^{t} = \sum_{j \in L} X_{kj}+C_{demandK} \hspace{4pt} \forall t \in T \hspace{4pt} \forall k \in L$ 
		\item $\sum_{i \in L} C_{DemandI}Y_I^t = \sum_{k \in J} G_k^t \hspace{4pt} \forall t \in T$
		\item $G_k^t \leq C_{GeneratorCapacityK} \hspace{4pt} \forall t\in T \hspace{4pt} \forall k \in J$
		\item $X_{ij}^t \leq C_{lineIJ}W_{ij}^t \hspace{4pt} \forall t \in T \hspace{4pt} \forall i,j \in P$
		\item $\sum_{i \in L} C_{RepairTimeI} F_{i}^t +\sum_{i,j \in P} C_{RepairTimeIJ} S_{ij}^t + \sum_{i \in L} \sum_{j<i \in L}  K_{ij}^t C_{TravelIJ} C_{broken} \leq 8 \hspace{4pt} \forall t \in T \hspace{4pt}$
		\item $Y_i \leq \sum_{0}^{t} F_i^t$ 
		\item $W_{ij} \leq \sum_{0}^{t} S_{ij}^t$
		\item $\sum_{j \in L} K_{0j}^t \geq 1$
		\item $\sum_{j \in L}K_{ij}^t - \sum_{j \in L}K_{ji}^t = 0  \hspace{4pt} \forall t \in T \hspace{4pt} \forall i \in L$
		\item A subtour elimination constraint 
	\end{enumerate}
	\subsubsection{Explanation of Constraint Systems}
	\begin{itemize}
		\item Constraint (1) defines flow balance equations for each node
		\item Constraint (2) defines input/output network balance. This is assuming Generator ramp time can be ignored, but that's fine since excess power can always be dropped to ground.
		\item Constraint (3) constrains power generation to be in the realm of feasible production
		\item Constraint (4) constrains line flow to be inside line capacity
		\item Constraint (5) constrains/decides what gets done during a shift
		\item Constraints (6) and (7) handle defining operations
		\item Constraints (8)-(10) handle the routing side of the problem
	\end{itemize}
	\subsubsection{Comments}
	\begin{itemize}
		\item I'm assuming that we're staying in the region of safe production for generators, a later thing to think about is "pushed" generators where they can be run in overdrive for short periods of time
		\end{itemize}
		
\end{document}