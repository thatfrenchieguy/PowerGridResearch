\documentclass{article}
\usepackage{amsmath}
\usepackage[fleqn]{mathtools}
\usepackage{amssymb}
\usepackage{amsthm}
\usepackage{enumitem}

\begin{document}
	\section{Introduction}
	\section{Roadmap}
	\begin{itemize}
		\item Exploratory Inventory Location Problem for Newsvendor pre-allocation
		\item contrived grid/contrived geography no-routing power grid repair scheduling with 1 crew
		\item contrived grid/contrived geography perfect information road repair with 1 crew
		\item inclusion of routing into power grid
		\item formulation of real geography and transition to GIS data
		\item transition from pipeflow power to DC grid model
		\item consult electrical engineers RE: power grid analysis
		\item generate real hurricane scenarios
		\item 1-2 response style interactions
		\item 1-2-1 revision style interactions
		\item joint solution
		\item incomplete information in roads
		
	\end{itemize}
	\section{Exploratary ILP}
	\section{Power Grid Models}
	\subsection{NonInteracting/NonRouting}
	\subsubsection{Variable Glossary}
	\begin{itemize}
		\item $C_{FlowIJ}$ is steady state flow on the grid before the hurricane
		\item $C_{lineIJ}$ is the capacity limit for the power line going from IJ
		\item $C_{RepairTimeI}$ is the time to repair node I
		\item $C_{demandI}$ is the power demand at location I in the pre-disaster steady state
		\item $C_{GeneratorCapacityK}$ is the maximum power generation for generator K
	\end{itemize}
	\subsubsection{Variables}
	\begin{itemize}
		
		\item $X_{ij}^{t}$ is the flow from i to j at time t
		\item $G_{k}^t$ is the production from generator k at time t
		\item $Y_i^t$ is 1 if node i is functioning at time t
		\item $W_{ij}^t$ is 1 if line ij is functioning at time t
		\item $F_i^t$ is 1 is node i is serviced at time t 
		\item $E_{ij}^t$ is 1 is line ij is serviced at time t	
	\end{itemize}
	\subsubsection{Sets}
	\begin{itemize}
		\item L is the set of nodes
		\item P is the set of power lines
		\item T is the planning horizon
		\item J is the set of Generators
	\end{itemize}
	\subsubsection{Model}
	$$	Minimize \sum_{i,j \in L} \sum_{t \in T} |C_{FlowIJ}-X_{ij}^t| $$
	
	Subject to:
	\begin{enumerate}[label=(\arabic*), leftmargin=*, itemsep=0.4ex, before={\everymath{\displaystyle}}]%
		
		\item $\sum_{i \in L} X_{ik}+G_{k}^{t} = \sum_{j \in L} X_{kj}+C_{demand\_k}Y_i^t \hspace{4pt} \forall t \in T \hspace{4pt} \forall k \in L$ 
		\item $\sum_{i \in L} C_{Demand\_i}Y_i^t = \sum_{k \in J} G_k^t \hspace{4pt} \forall t \in T$
		\item $G_k^t \leq C_{GeneratorCapacity\_k} \hspace{4pt} \forall t\in T \hspace{4pt} \forall k \in J$
		\item $X_{ij}^t \leq C_{lineIJ}W_{ij}^t \hspace{4pt} \forall t \in T \hspace{4pt} \forall i,j \in P$
		\item $\sum_{i \in L} C_{RepairTime\_i} F_{i}^t +\sum_{i,j \in P} C_{RepairTime\_ij} S_{ij}^t + \sum_{i \in L}  \leq 8 \hspace{4pt} \forall t \in T \hspace{4pt}$
		\item $Y_i^t \leq \sum_{0}^{t} F_i^t+initial \hspace{4pt} \forall i \in L\forall t \in T$ 
		\item $W_{ij}^t \leq \sum_{0}^{t} S_{ij}^t+initial \hspace{4pt} \forall i,j \in L\forall t \in T $
		\item $\sum_{t \in T}F_i^t \leq 1\hspace{4pt} \forall i \in L$
		\item $\sum_{t \in T}S_{ij}^t \leq 1\hspace{4pt} \forall i,j \in L$
		
	\end{enumerate}
	\subsubsection{Explanation of Constraint Systems}
	\begin{itemize}
		\item Constraint (1) defines flow balance equations for each node
		\item Constraint (2) defines input/output network balance. This is assuming Generator ramp time can be ignored, but that's fine since excess power can always be dropped to ground.
		\item Constraint (3) constrains power generation to be in the realm of feasible production
		\item Constraint (4) constrains line flow to be inside line capacity
		\item Constraint (5) constrains/decides what gets done during a shift
		\item Constraints (6) and (7) handle defining operations
		\end{itemize}
	\subsubsection{Comments}
	\begin{itemize}
		\item I'm assuming that we're staying in the region of safe production for generators, a later thing to think about is "pushed" generators where they can be run in overdrive for short periods of time
	\end{itemize}
	
	\subsection{NonInteracting/Routing}
	\subsubsection{Variable Glossary}
	\begin{itemize}
		\item $C_{FlowIJ}$ is steady state flow on the grid before the hurricane
		\item $C_{lineIJ}$ is the capacity limit for the power line going from IJ
		\item $C_{RepairTimeI}$ is the time to repair node I
		\item $C_{RepairTimeIJ}$ is the time to repair line IJ
		\item $C_{TravelIJ}$ is the travel time between nodes I and J
		\item $C_{broken}$ is a coefficient of "broken-ness" representing the average slowdown from debris on the road and minor flooding
		\item $C_{demandI}$ is the power demand at location I in the pre-disaster steady state
		\item $C_{GeneratorCapacityK}$ is the maximum power generation for generator K
	\end{itemize}
	\subsubsection{Variables}
	\begin{itemize}
		
		\item $X_{ij}^{t}$ is the flow from i to j at time t
		\item $G_{k}^t$ is the production from generator k at time t
		\item $Y_i^t$ is 1 if node i is functioning at time t
		\item $W_{ij}^t$ is 1 if line ij is functioning at time t
		\item $S_{ij}^t$ is 1 if line ij is serviced at time t
		\item $K_{ij}^t$ is 1 if node j follows node i in the tour at time t 
		\item $F_i^t$ is 1 is node i is serviced at time t 
		\item $E_{ij}^t$ is 1 is line ij is serviced at time t	
\end{itemize}
	\subsubsection{Sets}
	\begin{itemize}
		\item L is the set of nodes
		\item P is the set of power lines
		\item R is the set of roads
		\item T is the planning horizon
		\item J is the set of Generators
	\end{itemize}
	\subsubsection{Model}
	$$	Minimize \sum_{i \in L} \sum_{t \in T} C_{FlowIJ}-X_{ij}^t $$
	
	Subject to:
	\begin{enumerate}[label=(\arabic*), leftmargin=*, itemsep=0.4ex, before={\everymath{\displaystyle}}]%
		
		\item $\sum_{i \in L} X_{ik}+G_{k}^{t} = \sum_{j \in L} X_{kj}+C_{demandK} \hspace{4pt} \forall t \in T \hspace{4pt} \forall k \in L$ 
		\item $\sum_{i \in L} C_{DemandI}Y_I^t = \sum_{k \in J} G_k^t \hspace{4pt} \forall t \in T$
		\item $G_k^t \leq C_{GeneratorCapacityK} \hspace{4pt} \forall t\in T \hspace{4pt} \forall k \in J$
		\item $X_{ij}^t \leq C_{lineIJ}W_{ij}^t \hspace{4pt} \forall t \in T \hspace{4pt} \forall i,j \in P$
		\item $\sum_{i \in L} C_{RepairTimeI} F_{i}^t +\sum_{i,j \in P} C_{RepairTimeIJ} S_{ij}^t + \sum_{i \in L} \sum_{j<i \in L}  K_{ij}^t C_{TravelIJ} C_{broken} \leq 8 \hspace{4pt} \forall t \in T \hspace{4pt}$
		\item $Y_i^t \leq \sum_{0}^{t} F_i^t+initial \hspace{4pt} \forall i \in L$ 
		\item $W_{ij}^t \leq \sum_{0}^{t} S_{ij}^t+initial \hspace{4pt} \forall i,j \in L $
		\item $\sum_{j \in L} K_{0j}^t \geq 1 $
		\item $\sum_{j \in L}K_{ij}^t - \sum_{j \in L}K_{ji}^t = 0  \hspace{4pt} \forall t \in T \hspace{4pt} \forall i \in L$
		\item A subtour elimination constraint 
	\end{enumerate}
	\subsubsection{Explanation of Constraint Systems}
	\begin{itemize}
		\item Constraint (1) defines flow balance equations for each node
		\item Constraint (2) defines input/output network balance. This is assuming Generator ramp time can be ignored, but that's fine since excess power can always be dropped to ground.
		\item Constraint (3) constrains power generation to be in the realm of feasible production
		\item Constraint (4) constrains line flow to be inside line capacity
		\item Constraint (5) constrains/decides what gets done during a shift
		\item Constraints (6) and (7) handle defining operations
		\item Constraints (8)-(10) handle the routing side of the problem
	\end{itemize}
	\subsubsection{Comments}
	\begin{itemize}
		\item I'm assuming that we're staying in the region of safe production for generators, a later thing to think about is "pushed" generators where they can be run in overdrive for short periods of time
		\end{itemize}
	\subsection{DC Power Flow/Shortest Path "Routing"}	
	\subsubsection{Glossary}
	\begin{itemize}
		\item $C_{lineE}$ is the capacity limit for the power line E
		\item $C_{RepairTimeI}$ is the time to repair node I
		\item $C_{RepairTimeE}$ is the time to repair line E
		\item $C_{SP(i)}$ is the shortest path to node i from the central depot
		\item $C_{broken}$ is a coefficient of "broken-ness" representing the average slowdown from debris on the road and minor flooding
		\item $C_{demandI}$ is the power demand at location I in the pre-disaster steady state
		\item $C_{GeneratorCapacityK}$ is the maximum power generation for generator K
		\item $B_e$ is the line susceptance (imaginary part of admittance, also inverse of resistance) for power line e
	\end{itemize}
	\subsubsection{Variables}
	\begin{itemize}
		
		\item $X_{e}^{t}$ is the flow on line e at time t
		\item $G_{k}^t$ is the production from generator k at time t
		\item $V_i^t$ is 1 if node i is functioning at time t
		\item $W_{e}^t$ is 1 if line e is functioning at time t
		\item $S_{e}^t$ is 1 if line e is serviced at time t
		\item $F_i^t$ is 1 is node i is serviced at time t 
		\item $\theta_i^t$ is the phase angle for the power flow at i in time t
	
	\end{itemize}
	\subsubsection{Sets}
	\begin{itemize}
		\item N is the set of nodes
		\item E is the set of power lines
		\item R is the set of roads
		\item T is the planning horizon
		\item o(i) is the set of lines with origin i
		\item d(i) is the set of lines with destination i
		\item o(e) is the origin node of line e
		\item d(e) is the destination node of line e
	\end{itemize}
	\subsubsection{Model}
	$$	Minimize \sum_{i \in N} \sum_{t \in T} (1-W_i^t)*C_{demand\_i} $$
	
	Subject to:
	\begin{enumerate}[label=(\arabic*), leftmargin=*, itemsep=0.4ex, before={\everymath{\displaystyle}}]%
		
		\item $ X_e^t = B_e * (\theta_{origin}^t - \theta_{destination}^t) \forall t \in T \hspace{4pt} \forall e \in E$
		\item $ G_i^t - \sum_{l\|o(l)=i} X_l^t + \sum_{l\|d(l)=i} X_l^t = C_{demand\_i} \hspace{4pt} \forall t \in T \hspace{4pt} \forall i \in N$
		\item $G_k^t \leq C_{GeneratorCapacityK} V_{k}^t \hspace{4pt} \forall t\in T \hspace{4pt} \forall k \in N$
		\item $-C_{line\_e}W_{e}^t \leq X_{e}^t \leq C_{line\_e}W_{e}^t \hspace{4pt} \forall t \in T \hspace{4pt} \forall e \in E$
		\item $-C_{line\_e}V_{o(e)}^t \leq X_{e}^t \leq C_{line\_e}V_{o(e)}^t \hspace{4pt} \forall t \in T \hspace{4pt} \forall e \in E$
		\item $-C_{line\_e}V_{d(e)}^t \leq X_{e}^t \leq C_{line\_e}V_{d(e)}^t \hspace{4pt} \forall t \in T \hspace{4pt} \forall e \in E$
		
		\item $\sum_{i \in L} C_{RepairTimeI} F_{i}^t +\sum_{e \in E} C_{RepairTime\_e} S_{e}^t + \sum_{i \in L} F_i^t C_{SP(i)} + \sum_{e \in E} S_{e}^t * min(C_{SP(o(e))},C_{SP(d(e))}) \leq 8 \hspace{4pt} \forall t \in T \hspace{4pt}$
		\item $V_i^t \leq \sum_{0}^{t-1} F_i^t+initial \hspace{4pt} \forall i \in L$ 
		\item $W_{e}^t \leq \sum_{0}^{t-1} S_{e}^t+initial \hspace{4pt} \forall e \in E $
		\end{enumerate}
	\subsubsection{Explanation of Constraint Systems}
	\begin{itemize}
		\item Constraint (1) defines flow based on line limits and line susceptance as per Salmeron, Ross, and Baldick 2004
		\item Constraint (2) defines node power balance so that inflow has to match outflow at each node.
		\item Constraint (3) constrains power generation to be in the realm of feasible production conditional on the relevant node being operational
		\item Constraints (4)-(6) constrains line flow to be inside line capacity conditional on the relevant elements being operational
		\item Constraint (7) constrains/decides what gets done during a shift and handles shortest path travel time.
		\item Constraints (8) and (9) handle defining operations
	\end{itemize}
	\subsubsection{Comments}
	\begin{itemize}
		\item The assumption in this version of the model is that a vehicle can only do one operation per trip, so routing reduces to shortest path
		\item This also assumes DC power flow, which is a much more through version of power flow than pipeflow
		\item note that from constraint 9, once W is working, we can chose whether or not it's engaged
	\end{itemize}
\section{Road Models}
\subsection{Basic Routing Repair}
		\subsubsection{Glossary}
	\begin{itemize}
		\item $C_{ij}$ is a measure of the value of the road to relief supply delivery efforts
		\item $L_{ij}$ is the length of the road between nodes i and j when everything is working as normal
		\item $R_{ij}$ is the time to repair the road between i and j
	\end{itemize}
	\subsubsection{Variables}
	\begin{itemize}
		\item $X_{ij}^t$ is 1 if the road between nodes i and j is working at time t
		\item $S_{ij}^t$ is the length of the road between i and j at time t. 
		\item $K_{ij}^t$ is the decision variable that is 1 if j follows i in the tour at time t and 0 else.
		
	\end{itemize}
	\subsubsection{Sets}
	\begin{itemize}
		\item T is the set of time over the time horizon
		\item N is the set of nodes on the graph
	\end{itemize}
	\subsubsection{Model}
	$$	Minimize \sum_{t \in T} 1.02t \sum_{i,j \in N} C_{ij}*(1-X_{ij}^t) $$
	
	Subject to:
	\begin{enumerate}[label=(\arabic*), leftmargin=*, itemsep=0.4ex, before={\everymath{\displaystyle}}]%
		\item $\sum_{i,j \in N} S_{ij}^t K_{ij}^t \leq 8 \hspace{6pt} \forall t\in T$
		\item $S_{ij}^t = max(L_{ij},(1-X_{ij}^t)R_{ij}) \hspace{6pt} \forall t\in T \hspace{4pt} \forall i,j \in N$
		\item $\sum_{j \in N} K_{ij}^t - \sum_{j \in N} K_{ji}^t = 0 \hspace{6pt} \forall t\in T \hspace{4pt} \forall i \in N$
		\item $X_{ij}^t <= \sum_{v=0}^{t-1} K_{ij}^v + starting \hspace{6pt} \forall t\in T \hspace{4pt} \forall i,j \in N$
		\item $\sum_{i,j \in S; i\neq j} X_{ij}^t \leq |S|-1 \hspace{6pt} \forall S \subset N; \hspace{2pt} S \neq \emptyset$
	\end{enumerate}
	\subsubsection{Explanation of Constraint Systems}
	\begin{itemize}
		\item Constraint 1 is a scheduling constraint so that each tour has to be less than 8 hours of stuff
		\item Constraint 2 defines the length of a road to be either the travel length if it's working or the repair cost if it hasn't yet
		\item Constraint 3 is path connectivity for the tour
		\item Constraint 4 defines the functionality of each road. While it doesn't bind to 1, because there's a penalty for not being 1, it will choose 1 if possible
		\item Constraint 5 eliminates subtours to ensure a valid tour
	\end{itemize}
	\subsubsection{Comments}
	\begin{itemize}
		\item 
	\end{itemize}
		
\end{document}